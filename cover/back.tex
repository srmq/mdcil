\documentclass[11pt]{article}

\usepackage{url}
\usepackage{xcolor}
\usepackage{tikz}
\usepackage{mdframed}
\usepackage[papersize={7in,10in}, hmargin={.75in, .75in}, top=.25in]{geometry}

\definecolor{bg}{HTML}{00a092}
\pagecolor{bg}

\usepackage[T1]{fontenc}
\usepackage{newpxtext}
\usepackage[vvarbb,cmintegrals,cmbraces,bigdelims]{newpxmath}
\usepackage[scr=rsfso]{mathalfa}% \mathscr is fancier than \mathcal
\linespread{1.04}         % adds more leading (space between lines)
% quantifiers look strange, so change those back to normal:
	\DeclareSymbolFont{mysymbols}{OMS}{cmsy}{b}{n} %note we make the figures bold to better match newpx.  Replace the ``b'' with an ``m'' to undo this.
	%\SetSymbolFont{mysymbols}  {bold}{OMS}{cmsy}{b}{n}
	%\DeclareSymbolFont{myoperators}   {OT1}{cmr} {m}{n}
	%\SetSymbolFont{myoperators}{bold}{OT1}{cmr} {bx}{n}
	\DeclareMathSymbol{\forall}{\mathord}{mysymbols}{"38}
	\DeclareMathSymbol{\exists}{\mathord}{mysymbols}{"39}
	%\DeclareMathSymbol{\pm}{\mathbin}{mysymbols}{"06}
	%\DeclareMathSymbol{+}{\mathbin}{myoperators}{"2B}
	%\DeclareMathSymbol{-}{\mathbin}{mysymbols}{"00}
	%\DeclareMathSymbol{=}{\mathrel}{myoperators}{"3D}


\begin{document}

\pagestyle{empty}
~
\begin{center}
\begin{tikzpicture}[yshift=-.75in, scale=.9, remember picture, overlay, color=bg!80]
\def\r{.55}
\newcommand{\hexbox}[3]{
  \def\x{-cos{30}*\r*#1+cos{30}*#2*\r*2}
  \def\y{-\r*#1-sin{30}*\r*#1}
  \draw (\x,\y) +(90:\r) -- +(30:\r) -- +(-30:\r) -- +(-90:\r) -- +(-150:\r) -- +(150:\r) -- cycle;
  \draw (\x,\y) node{#3};
}


% Pascal's triangle
%put row of 1's down left side:
  \foreach \row in {0,...,16} {
    \hexbox{\row}{0}{\large 1}
  }
%fill in the rest of the triangle:
  \foreach \row in {1,...,16} {
    \pgfmathsetmacro{\entry}{1};
    \foreach \col in {1,...,\row} {
      % iterative formula : val = precval * (row-col+1)/col
      % (+ 0.5 to bypass rounding errors)
      \pgfmathtruncatemacro{\entry}{\entry*((\row-\col+1)/\col)+0.5};
      \global\let\entry=\entry
      \ifnum \entry<100
	\hexbox{\row}{\col}{\large \entry}
      \else \ifnum \entry<1000
	\hexbox{\row}{\col}{\entry}
      \else \ifnum \entry<10000
	\hexbox{\row}{\col}{\footnotesize \entry}
	\else
	\hexbox{\row}{\col}{\scriptsize \entry}
	\fi
      \fi
      \fi
    }
  }
  % \foreach \row in{17,...,20} {
  % \foreach \col in {1,...,\row} {
  % \hexbox{\row}{\col}{}
  % }
  % }
\end{tikzpicture}
\end{center}
\hspace{-3em}{\color{bg!10} Matemática}


\vskip 0.2in
%\noindent This gentle introduction to discrete mathematics is written for first and second year math majors, especially those who intend to teach.  The text began as a set of lecture notes for the discrete mathematics course at the University of Northern Colorado.  This course serves both as an introduction to topics in discrete math and as the ``introduction to proof'' course for math majors.  The course is usually taught with a large amount of student inquiry, and this text is written to help facilitate this.
\noindent O ano é 2020. Tínhamos começado mais um semestre da disciplina Matemática Discreta para o Curso de Sistemas da Informação do Centro de  Informática da UFPE. Tudo corria bem, continuaríamos usando como livro-texto da disciplina aquele que vínhamos adotando há alguns semestres, do qual temos vários exemplares na biblioteca que podem ser emprestados pelos alunos.

\vskip 1.5em

%\noindent Four main topics are covered: counting, sequences, logic, and graph theory.  Along the way proofs are introduced, including proofs by contradiction, proofs by induction, and combinatorial proofs.  The book contains over 470 exercises, including 275 with solutions and another 100 or so with hints.  Exercises range from elementary to quite challenging.
\noindent Então veio a pandemia. Os alunos estão privados do acesso à biblioteca. Mesmo que ela venha a reabrir, pode não ser seguro para muitos enfrentar o transporte público para acessá-la. As aulas passam para o modo online e o professor tem a obrigação de prover material online que possa ser utilizado de forma legal pelos alunos.

\vskip 1.5em

\noindent Na sua procura por material, o professor descobre que felizmente há alguns bons livros de matemática discreta disponíveis livremente na internet. Mas não sem alguns inconvenientes:
\begin{itemize}
\item[--] O material está em inglês. Embora seja obrigação do estudante de computação ter um bom nível nesta língua, muitos estudantes só descobrem isso ao ingressar na universidade. Nossa disciplina de matemática discreta é ensinada no primeiro período. Adotar material em inglês colocaria uma grande barreira de acesso à disciplina para vários estudantes que não tiveram a oportunidade de se preparar bem em inglês antes de ingressar na universidade.
\item[--] É difícil de achar um material online que cubra todos os tópicos que costumamos ver na disciplina.
\item[--] Alguns materiais são voltados para cursos de matemática ou com enfoque maior em informática teórica do que adotamos no curso de Sistemas da Informação.
\end{itemize}

\vskip 1em

\noindent Sendo assim, esse livro se propõe a adaptar diferentes fontes disponíveis sob licenças livres, de maneira a fornecer um material que possa ser usado livremente para ensinar a disciplina de Matemática Discreta, comumente ofertada nos primeiros semestres dos cursos de computação, em países de língua portuguesa.  

\vskip 5em
\begin{center}Para acessar a versão atual, bem como para obter o código-fonte do livro, visite: \\
	% \includegraphics[width=1in]{qrcode.png}\\
	\url{http://discrete.openmathbooks.org/}.
\end{center}


\clearpage

\end{document}
